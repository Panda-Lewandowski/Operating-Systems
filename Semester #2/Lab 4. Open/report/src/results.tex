% !TEX root = main.tex
\section{Вывод}

Исходя из вышеприведенных рассуждений, можно сделать несколько выводов.

\begin{enumerate}
	\item  Предпочтительней использовать функцию $fopen()$, т.к. $fopen()$ выполняет ввод-вывод с буферизацией, что может оказаться значительно быстрее, чем  с использованием $open()$, $FILE *$ дает возможность использовать $fscanf()$ и другие функции $stdio.h$.
	\item Следует помнить о буферизации и вовремя использовать $fclose()$ для записи в файл.
	\item С острожностью использовать $fflush()$, т.к. она оставляет поток открытым.
	\item Необходимо следить за режимом, с котором открывается поток. 
	\item Созданный новый дескриптор открытого файла  изначально не разделяется с любым другим процессом, но разделение может возникнуть через $fork()$.
	\item Функции $fscanf, fprintf, fopen, fclose$ являются обертками высшего уровня над системными вызовами $open, close, read, write$.
\end{enumerate}